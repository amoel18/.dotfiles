\documentclass[danish,a4paper,11pt]{article}
\usepackage{lmodern}
% Some basic packages
\makeatletter
\def\input@path{{../../../}}
\makeatother
\input{Preamble.tex}
\title{Non-linear parabolic pdes in mathematical finance (Black Scholes)}
\author{Anders Kinch Møller}
\date{}
\begin{document}
\maketitle
  Relaxing the assumptions of the BS-model results in a non-linear BS equation. \\
  For example we could look at the volatility \( \sigma \) and the drift \( \mu  \) depending on time \( t \) or the transaction cost not being \( 0 \). In the case of transaction cost, we would need to relax the hedging condition. This can be done by trading at discrete times, as we see in the Leland's model.\\

  stochastic volatility (Heston model)\\
  stochastic interest rates \\
  increase number of underlying assets\\
  look at different payoff functions (like options on the maximum or average)\\
  \\
  Hamilton-Jacobi-Bellman equations \\
  allowing for jumps in the development of the assets (integro PDE's) \\

  A pde is said to be parabolic if \( B^{2} -4AC = 0 \) \\
  considering stability we find a more telling constraint on the size of \( \delta t  \). Heat conduction is said to be stable if the \( u_{j}^{n}  \) do not grow in magnitude with \( n \) . \\
In general, the PDEs of relevance are of theconvection-di usion-reactiontype innspace variables and one time variable.\\
Look at a generalized model, we relax the assumptions of the BS-model: we now model a multi=asset environment with dividend \( D \). \\
We want to find a numerical scheme which produce accurate results. \\
In order to reduce the number of solutions for the pde, we define the initial condition and boundary conditions. These are:
\begin{itemize}
  \item First boundary value problem (Dirchlet problem)
  \item Second boundary value problem (Neumann, robin problems).
  \item Cauchy problem.

\end{itemize}

Assumptions:
\begin{enumerate}
  \item The underlying stock \( S(t) \) follows a geometric brownian motion.
\end{enumerate}
\begin{align*}
  dS  &= \mu S dt  + \sigma S dW && \text{(geometric brownian motion)}  \\
  V & = V(T-t,S_t;r, \sigma, K) \\
  dV    & = \frac{\partial V}{\partial t} dt  + \frac{\partial V}{\partial S} dS  + \frac{1}{2} \frac{\partial^{2} V}{\partial S^{2}} dS^{2} \text{By Itos lemma}
\end{align*}
\end{document}
