% #copy the following comment into tex file
\documentclass[12pt, twoside,a4paper]{article}
\usepackage{/home/i/preamble}
\newcommand\m[1]{\begin{pmatrix}#1\end{pmatrix}} 
% \usepackage{natbib}
% \bibliographystyle{plain}
% \usepackage[english]{babel}
% \begin{filecontents*}{Bibliografyy.bib}
%\end{filecontents}
% \begin{filecontents*}{mybib.bib}
% @article{key,
% author={},
% title={}
% .
% .
% .
% }
% \end{filecontents*} 

% \usepackage[numbers]{natbib}
\usepackage{url}
\title{MM852}
\author{Anders Kinch}
\date {}

\pagestyle{fancy}
\fancyhead[C]{\rule{.5\textwidth}{4\baselineskip}}% Add something BIG in the header
\fancyhf{}
\pdfminorversion=7


\begin{document}


\maketitle


Assignment
Assume that an underlying follows the Heston model with parameters $$\nu_0=0.04, S_0=100, \xi=0.3, \rho=-0.5, \mu=0.1,
\kappa = 1.2, \text{and} \theta=0.04. $$ Using the Multilevel Monte Carlo method, determine the fair price of



a) a European call option,

\begin{equation}
  f_E (X) = \max \{0, X(T)-K\}
\end{equation}


b) a European put option,

based on the above asset and a strike price of 100, time to expiration equal 1 and riskless interest rate equal 5%.

Explain the Heston model and how you simulate it, as well as the Multilevel Monte Carlo method. Document simulation results,  a sufficient accuracy, the code and how efficient your simulation is (including showing that variances and computational effort behave as expected), and what you gain by the multilevel Monte Carlo method (comparing with Standard Monte Carlo and Standard Monte Carlo with Control Variates). Discuss also how you efficiently can use Control Variates in combination with the Multilevel Monte Carlo method, and document by corresponding simulations.

The payoff function for a European call option is given by  


% \cite{latexcompanion,knuthwebsite}. 
\medskip

\nocite{*}
\bibliography{/home/i/tex/tt/mybib.bib}
%\bibliographystyle{plain}

\pdfsuppresswarningpagegroup=1


\end{document}



