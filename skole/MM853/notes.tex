\documentclass[12pt]{article}
% Some basic packages
\usepackage[utf8]{inputenc}
\usepackage[T1]{fontenc}
\usepackage{textcomp}
\usepackage[danish]{babel}
\usepackage{graphicx}
\usepackage{float}
\usepackage{enumitem}

\pdfminorversion=7

% Don't indent paragraphs, leave some space between them
\usepackage{parskip}

% Hide page number when page is empty
\usepackage{emptypage}
\usepackage{subcaption}
\usepackage{multicol}
\usepackage{xcolor}

% Other font I sometimes use.
% \usepackage{cmbright}

% Math stuff
\usepackage{amsmath, amsfonts, mathtools, amsthm, amssymb}
% Fancy script capitals
\usepackage{mathrsfs}
\usepackage{cancel}
% Bold math
\usepackage{bm}
% Some shortcuts
\newcommand\N{\ensuremath{\mathbb{N}}}
\newcommand\R{\ensuremath{\mathbb{R}}}
\newcommand\Z{\ensuremath{\mathbb{Z}}}
\renewcommand\O{\ensuremath{\emptyset}}
\newcommand\Q{\ensuremath{\mathbb{Q}}}
\newcommand\C{\ensuremath{\mathbb{C}}}

% Easily typeset systems of equations (French package)
\usepackage{systeme}

% Put x \to \infty below \lim
\let\svlim\lim\def\lim{\svlim\limits}

%Make implies and impliedby shorter
\let\implies\Rightarrow
\let\impliedby\Leftarrow
\let\iff\Leftrightarrow
\let\epsilon\varepsilon

% Add \contra symbol to denote contradiction
\usepackage{stmaryrd} % for \lightning
\newcommand\contra{\scalebox{1.5}{$\lightning$}}

% \let\phi\varphi

% Command for short corrections
% Usage: 1+1=\correct{3}{2}

\definecolor{correct}{HTML}{009900}
\newcommand\correct[2]{\ensuremath{\:}{\color{red}{#1}}\ensuremath{\to }{\color{correct}{#2}}\ensuremath{\:}}
\newcommand\green[1]{{\color{correct}{#1}}}

% horizontal rule
\newcommand\hr{
    \noindent\rule[0.5ex]{\linewidth}{0.5pt}
}

% hide parts
\newcommand\hide[1]{}

% si unitx
\usepackage{siunitx}
\sisetup{locale = FR}

% Environments
\makeatother
% For box around Definition, Theorem, \ldots
%\mdfsetup{skipabove=1em,skipbelow=0em}
%\theoremstyle{definition}
\theoremstyle{plain}
\newtheorem*{definition*}{Definition}
\newtheorem{definition}{Definition}
% End example and intermezzo environments with a small diamond (just like proof
% environments end with a small square)
\usepackage{etoolbox}
\AtEndEnvironment{vb}{\null\hfill$\diamond$}%
\AtEndEnvironment{intermezzo}{\null\hfill$\diamond$}%
% \AtEndEnvironment{opmerking}{\null\hfill$\diamond$}%

% Fix some spacing
% http://tex.stackexchange.com/questions/22119/how-can-i-change-the-spacing-before-theorems-with-amsthm
\makeatletter
\def\thm@space@setup{%
  \thm@preskip=\parskip \thm@postskip=0pt
}


% Exercise
% Usage:
% \oefening{5}
% \suboefening{1}
% \suboefening{2}
% \suboefening{3}
% gives
% Oefening 5
%   Oefening 5.1
%   Oefening 5.2
%   Oefening 5.3
\newcommand{\oefening}[1]{%
    \def\@oefening{#1}%
    \subsection*{Oefening #1}
}

\newcommand{\suboefening}[1]{%
    \subsubsection*{Oefening \@oefening.#1}
}


% \lecture starts a new lecture (les in dutch)
%
% Usage:
% \lecture{1}{di 12 feb 2019 16:00}{Inleiding}
%
% This adds a section heading with the number / title of the lecture and a
% margin paragraph with the date.

% I use \dateparts here to hide the year (2019). This way, I can easily parse
% the date of each lecture unambiguously while still having a human-friendly
% short format printed to the pdf.

\usepackage{xifthen}
\def\testdateparts#1{\dateparts#1\relax}
\def\dateparts#1 #2 #3 #4 #5\relax{
    \marginpar{\small\textsf{\mbox{#1 #2 #3 #5}}}
}

\def\@lecture{}%
\newcommand{\lecture}[3]{
    \ifthenelse{\isempty{#3}}{%
        \def\@lecture{Lecture #1}%
    }{%
        \def\@lecture{Lecture #1: #3}%
    }%
    \subsection*{\@lecture}
    \marginpar{\small\textsf{\mbox{#2}}}
}



% These are the fancy headers

% LE: left even
% RO: right odd
% CE, CO: center even, center odd
% My name for when I print my lecture notes to use for an open book exam.
% \fancyhead[LE,RO]{Gilles Castel}





% Todonotes and inline notes in fancy boxes
% \usepackage{todonotes}
% \usepackage{tcolorbox}

% Make boxes breakable
% \tcbuselibrary{breakable}

% Verbetering is correction in Dutch
% Usage:
% \begin{verbetering}
%     Lorem ipsum dolor sit amet, consetetur sadipscing elitr, sed diam nonumy eirmod
%     tempor invidunt ut labore et dolore magna aliquyam erat, sed diam voluptua. At
%     vero eos et accusam et justo duo dolores et ea rebum. Stet clita kasd gubergren,
%     no sea takimata sanctus est Lorem ipsum dolor sit amet.
% \end{verbetering}
\newenvironment{verbetering}{\begin{tcolorbox}[
    arc=0mm,
    colback=white,
    colframe=green!60!black,
    title=Opmerking,
    fonttitle=\sffamily,
    breakable
]}{\end{tcolorbox}}

% Noot is note in Dutch. Same as 'verbetering' but color of box is different
\newenvironment{noot}[1]{\begin{tcolorbox}[
    arc=0mm,
    colback=white,
    colframe=white!60!black,
    title=#1,
    fonttitle=\sffamily,
    breakable
]}{\end{tcolorbox}}




% Figure support as explained in my blog post.
\usepackage{import}
\usepackage{xifthen}
\usepackage{pdfpages}
\usepackage{transparent}
\newcommand{\incfig}[1]{%
    \def\svgwidth{\columnwidth}
    \import{./figures/}{#1.pdf_tex}
}


  \usepackage[
    backend=biber,
    style=phys,
  ]{biblatex}

 \addbibresource{References}
 \usepackage{csquotes}

% Fix some stuff
% %http://tex.stackexchange.com/questions/76273/multiple-pdfs-with-page-group-included-in-a-single-page-warning
\pdfsuppresswarningpagegroup=1



% My name
\title{A very Simple}
\author{Anders Kinch}
\date{\today}

\begin{document}
\maketitle

\section{}

\begin{definition*} 
  1.1 Fields
\end{definition*}

\begin{definition*} 
 1.2 Vector spaces 
\end{definition*}

\begin{definition*} 
  1.4 Subspaces 
\end{definition*}


\begin{definition} 
 1.5 Linear combination 
\end{definition}


\begin{definition} 
 1.7 Span 
\end{definition}


\begin{definition} 
 1.9 Linear dependence 
\end{definition}



\begin{definition} 
 1.12 Basis 
\end{definition}


\begin{definition} 
 1.14 Coordinate 
\end{definition}

\section{2. Linearity}


\begin{definition} 
 2.1 Linear map 
\end{definition}


\begin{definition} 
 2.4 Null-space and Range 
\end{definition}

\begin{definition} 
 2.6 Isomorphisms
\end{definition}

\begin{definition} 
 2.13 Quotient space  
\end{definition}


\begin{definition} 
 2.15 Quotient map 
\end{definition}



\begin{definition} 
  2.16 Invariant 
\end{definition}


\begin{definition} 
 2.18 Nullity and rank 
\end{definition}


\begin{definition} 
 2.21 Matrix 
\end{definition}


\begin{definition} 
 2.23 Product 
\end{definition}


\begin{definition} 
  2.24 Algebra (bilinear map) 
\end{definition}

\section{3. Duality}


\begin{definition} 
 3.1 Dual space 
\end{definition}



\begin{definition} 
 3.3 i'th coordinate functional 
\end{definition}



\begin{definition} 
 3.7 Annihilator  
\end{definition}


\begin{definition} 
 3.12 Adjoint  
\end{definition}


\begin{definition} 
 3.19 Double dual 
\end{definition}


\begin{definition} 
 3.21 Natural correspondence 
\end{definition}

\section{4. Bilinear maps}

\begin{definition} 
  4.1 Bilinear map
\end{definition}


\begin{definition} 
 4.3 Multilinear map 
\end{definition}


\begin{definition} 
 4.6 Symmetry and skew-symmetry 
\end{definition}



\begin{definition} 
 4.7 Alternating 
\end{definition}


\begin{definition} 
 4.9 Quadratic form 
\end{definition}


\begin{definition} 
 4.11 Orthogonal 
\end{definition}


\begin{definition} 
 4.14 non-degenerate 
\end{definition}


\begin{definition} 
 4.15 Symplectic    
\end{definition}


\begin{definition} 
 4.17 Positive and negative definite 
\end{definition}


\begin{definition} 
 4.19 Quadrics and conics 
\end{definition}


\section{5. Sums and products}

\begin{definition} 
  5.1 Direct sum
\end{definition}



\begin{definition} 
 5.5 Complement 
\end{definition}


\begin{definition} 
  5.9 (external) Direct sum 
\end{definition}



\begin{definition} 
 5.10 Projection 
\end{definition}


\begin{definition} 
 5.12 idempotent 
\end{definition}


\begin{definition} 
 5.14 Tensor product 
\end{definition}


\begin{definition} 
 5.18 Pair  
\end{definition}


\section{6. Eigendecomposition}


\begin{definition} 
 6.1 Eigenspace 
\end{definition}



\begin{definition} 
  6.2 Spectrum and (geometric) multiplicity 
\end{definition}



\begin{definition} 
 6.3 Polynomial 
\end{definition}



\begin{definition} 
 6.7 Minimal polynomail
\end{definition}


\begin{definition} 
 6.10 Flag 
\end{definition}


\begin{definition} 
 6.11 Triangular matrix 
\end{definition}


\begin{definition} 
  6.15 Reduce
\end{definition}


\begin{definition} 
  6.17 Direct sum
\end{definition}


\section{7. Generalized Eigendecomposition}

\begin{definition} 
  7.1 Generalized eigenstate
\end{definition}



\begin{definition} 
 7.2 algebraic multiplicity 
\end{definition}



\begin{definition} 
 7.3 nilpotent 
\end{definition}


\section{8. Orthogonality}



\begin{definition} 
 8.1 Inner product  
\end{definition}



\begin{definition} 
 8.2 Length and orthogonal 
\end{definition}



\begin{definition} 
 8.8 Orthogonal Direct sum
\end{definition}



\begin{definition} 
 8.10 Orthogonal projection
\end{definition}


\begin{definition} 
 8.15 Adjoint 
\end{definition}



\begin{definition} 
 8.18 self-adjoint 
\end{definition}



\begin{definition} 
 8.21 Mutually orthogonal 
\end{definition}

 
 \begin{definition} 
  8.24 Unitary 
 \end{definition}

 
 \begin{definition} 
  8.25 Trace 
 \end{definition}


 
 \begin{definition} 
  8.28 Trace of a linear map 
 \end{definition}

 \section{9. Spectral theorems}


 
 \begin{definition} 
  9.1 orthogonally diagnoable  
 \end{definition}


 
 \begin{definition} 
  9.6 Normal map 
 \end{definition}

 
 \begin{definition} 
  9.14 Functions of maps 
 \end{definition}

 \section{Determinants}
 
 \begin{definition} 
  10.1 Alternating forms
 \end{definition}



\end{document}




